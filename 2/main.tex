\documentclass[14pt,a4paper]{article}
\usepackage[utf8]{inputenc}
\usepackage[russian]{babel}
\usepackage{amsfonts}
\usepackage{amsmath}
\usepackage{amsthm}
% \usepackage{showframe}

\title{Операторные уравнения для обобщённых управляемых систем}

\theoremstyle{plain}
\newtheorem{theorem}{Теорема}[section]
\newtheorem{lemma}{Лемма}[section]
\newtheorem{corollary}{Следствие}[lemma]

\numberwithin{equation}{section}

\begin{document}

\begin{abstract}
    Классическая математическая модель описывающая процессы движения одномерной управляемой системы с одной
    степенью свободы через обыкновенные дифференциальные уравнения, обобщаются на случай обобщённых операторных
    уравнений в банаховом пространстве. Устанавливается их корректная разрешимость по Адамару с применением
    методов теории сильно непрерывных полугрупп линейных операторов.

    Результаты применяются к широкому классу математических моделей описываемых уравнениями через
    операторные ортогональные многочлены.

    \textbf{Ключевые слова}: управляемые системы, операторные уравнения, корректные задачи,
    сильно непрерывные полугруппы, операторные ортогональные многочлены.
\end{abstract}

\maketitle
\tableofcontents

\section{Введение}

% TODO: 8
Как известно (см. 8), уравнение
\begin{equation}
    \label{eq:1}
    \sum_{k=0}^{n} a_m \frac{d^k u(t)}{dt^k} = \sum_{z=0}^{m} b_z \frac{d^z Z(t)}{dt^z}, \quad m < n
\end{equation}
где $u(t)$ -- искомая функция (сигнал на выходе системы), $Z(t)$ -- заданная функция (сигнал на входе системы),
$t \in (t_0, \infty)$, относится к классу дифференциальных уравнений описывающих процессы в замкнутой
управляемой системе.

Классическим методом решения таких уравнений является метод интегрального преобразования Лапласа.

В настоящей заметке уравнение \ref{eq:1} обобщается на случай операторных уравнений в следующей постановке:
пусть $E$ -- банахово пространство и $A$ -- линейный, регулярный, т.е. имеющий хотя бы для одного $t_0 \in \mathbb{C}$
ограниченный обратный $(\lambda_0^I - A)^{-1}$, $I$ -- тождественный оператор.

% TODO: 5
В соответствии с [5] для такого оператора область определения степеней $A^n$, $D(A^n)$ превращается в
банахово пространство $\mathbb{D}_n$ относительно нормы
\begin{equation}
    \|u\|_n = \sum_{k=0}^{n} \|A^k u\|_E.
\end{equation}

Для $f \in \mathbb{D}_m$ рассматривается операторное уравнение
\begin{equation}
    \label{eq:2}
    \mathbb{P}_n(A)u = \sum_{k=0}^{n}a_k A^k u = \sum_{z=0}^{m} b_z A^z f = \mathbb{Q}_m(A)f,
\end{equation}
$m < n, a_k, b_z \in \mathbb{C}$, $\mathbb{C}$ -- комплексная плоскость.

Ищется элемент $u \in \mathbb{D}_n$, удовлетворяющий уравнению \ref{eq:2} и указываются условия
на исходные данные при которых задача \ref{eq:2} корректно разрешима с оценкой
\begin{equation}
    \|u\|_m \le M_m \|f\|_m.
\end{equation}

\section{Необходимые факты и понятия}

\subsection{Корректность по Адамару}

Пусть $E$ -- банахово пространство с нормой $\|\cdot\|_E$ и $A$ линейный замкнутный оператор с областью
определения $D(A) \subset E$.

Задача нахождения решения $u$ уравнения
\begin{equation}
    \label{eq:3}
    Au = f,
\end{equation}
для $f \in E$ и $u \in D(A)$ называется корректной по Адамару, если:
\begin{enumerate}
    \item существует единственное $u \in D(A)$, удовлетворяющее \ref{eq:3};
    \item для всякого $f \in E$ выполянется неравенство
    \begin{equation}
        \label{eq:4}
        \|u\|_E \le M \|f\|_E.
    \end{equation}
    $M$ - не зависит от $f$;
\end{enumerate}
Неравенство \ref{eq:4} означает, что оператор $A$ имеет ограниченный в $E$ обратный оператор $A^{-1}$ с оценкой
\begin{equation}
    \label{eq:5}
    \|A^{-1}f\| \le M \|f\|.
\end{equation}
Неравенство \ref{eq:5} называется неравенством корректности.

\subsection{Сильно непрерывные полугруппы}

% TODO: 3
Говорят (3), что семейство $V(t), (t \ge 0)$ ограниченных в $E$ операторов образует сильно непрерывную
полугруппу (полугруппу класса $C_0$), если выполнены следующие условия:
\begin{enumerate}
    \item $V(0)\varphi = \varphi, \quad \varphi \in E$;
    \item $V(t+s)\varphi = V(s)U(t)\varphi, \quad \varphi \in E$;
    \item $\lim\limits_{t\to0} \|V(t)\varphi-\varphi\| = 0, \quad \varphi \in E$;
    \item $\lim\limits_{h\to0} \|\frac{1}{h}[V(h)\varphi-\varphi]\| = A\varphi, \quad \varphi \in D(A)$.
\end{enumerate}
Отсюда
\begin{equation}
    \frac{dV(t)}{dt}\bigg|_{t=0} \varphi = A\varphi.
\end{equation}

Для всякой полугруппы класса $C_0$ существуют константы $M, \omega$ при которых выполняется оценка
\begin{equation}
    \label{eq:6}
    \|V(t)\| \le Me^{\omega t}, \quad t \ge 0.
\end{equation}
$\omega$ -- порядок роста полугруппы.

% В дальнейшем такие полугруппы с производящим оператором $A$ (генератором) обозначаются через $V(t, A)$.

% TODO: x3
Если в \ref{eq:6} $\omega \le 0$, то такие полугруппы по Балакришнану (x3) называются равностепенно
непрерывными и их производящий оператор $A$ обладает тем свойством, что для $\alpha \in (0, 1)$ существуют
дробные степени оператора $-A$, имеющие вид
\begin{equation}
    (-A)^\alpha \varphi = \frac{1}{\Gamma(-\alpha)}
\int\limits_0^\infty \frac{[V(t, -A)\varphi-\varphi]}{t^{1+\alpha}} dt
\end{equation}

При этом оператор $-(-A)^\alpha$ является генератором полугруппы $V(t, -(-A)^\alpha)\varphi$ вида
\begin{equation}
    V(t, -(-A)^\alpha)\varphi = \int\limits_0^\infty h_{\alpha,t}(\xi) V(\xi, -A) \varphi d\xi,
\end{equation}
% TODO: x3, 3
где $h_{\alpha,t}(\xi)$ -- функция Иосиды (x3). Отсюда в силу [3], определены для $0 < \alpha \le n$
все степени $(-A)^\alpha$.

% TODO: 7
Применяя этот факт, в [7] рассматривается операторное уравнение
\begin{equation}
    \label{eq:7}
    \sum_{k=0}^{n} a_k A^k u = f, \quad f \in E
\end{equation}
и приводится следующий результат: если все корни многочлена $P_n(x), x_k$ принадлежат резольвентному множеству
оператора $A$, то уравнение \ref{eq:7} имеет единственное решение $u \in D(A^n)$ и оно представимо в виде
\begin{equation}
    u = \int\limits_0^\infty q(t) V(t, A)fdt,
\end{equation}
где функция $q(t)$ является решением задачи
\begin{equation}
    \label{eq:14}
    \sum_{k=0}^{n} (-1)^k a_k q^{(k)}(t) = \delta(t);
\end{equation}
\begin{equation}
    q(0) = q'(0) = ... = q^{(n-1)}(0) = 0;
\end{equation}
$\delta(t)$ -- дельта функция.

\subsection{Формулировка результата}

Будем рассматривать уравнение \ref{eq:2} с оператором $A$ таким, что оператор $-A$ является генератором
$C_0$-полугруппы удовлетворяющей оценке
\begin{equation}
    \|V(t, -A)\varphi\| \le Me^{-\omega t}, quad \omega > 0,
\end{equation}
и, следовательно, по Балакришнану оператор $A$ имеет все степени $A^m, m \le n$.

Основным результатом является.

\begin{theorem}
    Если корни $\nu_k$ многочлена $P_n(x)$ взаимно простые и удовлетворяют условию
    \begin{equation}
        \label{eq:13}
        Re \nu_k < \omega, \quad k = 1...n,
    \end{equation}
    % TODO: 1.11 -- x3
    а оператор $-A$ является генератором полугруппы $V(t, -A)$ с оценкой (1.11?), то уравнение
    \ref{eq:2} имеет единственное решение $u \in \mathbb{D}_n$, оно имеет вид
    \begin{equation}
        u = \int\limits_0^\infty \sum_{z=0}^{m} q^{(z)}(t) V(t, -A) fdt = \\
        \int\limits_0^\infty Q_m (t) V(t, -A)fdt,
    \end{equation}
    где $q(t)$ является решением задачи
    \begin{equation}
        \label{eq:8}
        \sum_{k=0}^{n} a_n q^{(k)}(t) = \delta(t);
    \end{equation}
    \begin{equation}
        \label{eq:9}
        q(0) = q'(0) = ... = q^{(n-1)}(0) = 0;
    \end{equation}
    и справедлива оценка корректности
    \begin{equation}
        \label{eq:15}
        \|u\|_m \le M_m \|f\|_m, \\
    \end{equation}
    \begin{equation*}
        \|f\|_m = \sum_{k=0}^{m} \|A^k f\|
    \end{equation*}
\end{theorem}

\subsection{Доказательство теоремы}

% TODO: 1
В силу простоты корней многочлена $P_n(x)$ применение преобразования Лапласа к задаче \ref{eq:8}-\ref{eq:9}
даёт её решение [1]
\begin{equation}
    \label{eq:10}
    q(t) = \sum_{k=0}^{n} \frac{e^{\nu_{k}t}}{P'_n(\nu_k)}
\end{equation}

Из \ref{eq:10} следует неравенство
\begin{equation}
    \label{eq:11}
    |q(t)| \le \sum_{k=0}^{n} \frac{e^{Re \nu_k t}}{P'_n(\nu_k)} \le M_n e^{\overline{\nu} t},
\end{equation}
$\overline{\nu} = max(\nu_k)$

Из \ref{eq:11} и \ref{eq:9} для $f \in D(A)$ следует равенство
\begin{equation}
    \label{eq:12}
    \int\limits_0^\infty q(t) A^k V(t, -A)f = \int\limits_0^\infty q^{(k)}(t) V(t, -A)fdt,
\end{equation}
которое можно доказать по индукции, так как при $k=1$ имеем при $\overline{\nu} < \omega$, интегрируя по частям
\begin{gather*}
    \int\limits_0^\infty q(t) A V(t, -A)fdt = \\
    = -\int\limits_0^\infty q(t) (-A V(t, -A))fdt = \\
    = -\int\limits_0^\infty q(t) \frac{dV(t, -A)}{dt}fdt = \\
    = -q(t) V(t, -A)\bigg|_{t=0}^{\infty}f + \int\limits_0^\infty q(x) V(t, -A)fdt
\end{gather*}
% TODO: ? ??
И так как $|q(t) V(t, -A)f \le Me^{(\nu \omega t)}$, то и $q(0)=0$, то получаем (?) для $?? = 1$.

Применяя к уравнению \ref{eq:2} формулы \ref{eq:13} и \ref{eq:12}, получаем представление для решения
\begin{equation}
    u = \int\limits_0^\infty q(t) Q_m(A) V(t, -A)fdt = \int\limits_0^\infty Q_m (t) V(t, -A)fdt.
\end{equation}
Из которого следует равенство
\begin{equation}
    A^k u = \int\limits_0^\infty Q_m (t) V(t, -A) A^k fdt
\end{equation}

Используя соотношения \ref{eq:14}, \ref{eq:11}, \ref{eq:12} после стандартных оценок получаем неравенство
\begin{gather*}
    \|A^k u\| \le \int\limits_0^\infty |Q_m (t)| \|V(t, -A) A^k f\| dt \le \\
    \le M_m \int\limits_0^\infty e^{(\overline{\nu} - \omega)t} at \cdot \|A^k f\| = \\
    = \frac{M_m \|A^k f\|}{\omega - \overline{\nu}}
\end{gather*}
Суммируя по $k = 1, 2 ... m$, получаем оценку \ref{eq:15} и доказательство теоремы.

\section{Заключение}

В заключение заметим, что в классы многочленов с простыми корнями, рассмотренными в настоящей работе,
попадают ортогональные многочлены, что характеризует широту возможных приложений полученного результата.
Кроме того этому также способствует и многообразие примеров операторов производящих сильно непрерывные
% TODO: 7
полугруппы например приведённых в [7].

\clearpage

\begin{thebibliography}{99}
    % \bibitem{Krein} {\it Крейн~С.Г.} Линейные дифференциальные уравнения в банаховом пространстве "--- М.: Наука, 1967.
\end{thebibliography}

\end{document}
